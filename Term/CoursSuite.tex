% Created 2021-09-27 lun. 20:18
% Intended LaTeX compiler: pdflatex
\documentclass[11pt]{article}
\usepackage[utf8]{inputenc}
\usepackage[T1]{fontenc}
\usepackage{graphicx}
\usepackage{grffile}
\usepackage{longtable}
\usepackage{wrapfig}
\usepackage{rotating}
\usepackage[normalem]{ulem}
\usepackage{amsmath}
\usepackage{textcomp}
\usepackage{amssymb}
\usepackage{capt-of}
\usepackage{hyperref}
\usepackage{minted}
\author{Delhomme Fabien}
\date{\today}
\title{Cours Suite}
\hypersetup{
 pdfauthor={Delhomme Fabien},
 pdftitle={Cours Suite},
 pdfkeywords={},
 pdfsubject={},
 pdfcreator={Emacs 27.2 (Org mode 9.5)}, 
 pdflang={English}}
\begin{document}

\maketitle
\section{Suites arithmétiques et géométriques : définition et techniques}
\label{sec:orge4e8058}
\subsection{Suite arithmétiques}
\label{sec:org81635d1}
\begin{prop}
On dit qu'une suite est une suite arithmétique si chaque terme de la suite est égal au terme précédent, ajouté d'un même nombre $r$
\end{prop}

\begin{ex}
    Si j'ajoute $3$ pour passer d'un terme à un autre à chaque fois, ma suite est arithmétique, de raison $3$.
\end{ex}

\begin{ex}
\begin{center}
\begin{tabular}{lrrrrr}
\hline
n & 0 & 1 & 2 & 3 & 4\\
\hline
u\textsubscript{n} & 0 & 3 & 6 & 9 & 12\\
\hline
\end{tabular}
\end{center}
\end{ex}

\begin{prop}
    Pour trouver la raison d'une suite arithmétique, il suffit de calculer la différence de deux termes consécutifs
\end{prop}

\begin{ex}
Dans la suite donnée en exemple, si on calcule, par exemple :
\[
u_3 - u_2 = 9 - 6 = 3
\]
on retrouve bien la raison de la suite, $r=2$
\end{ex}

\begin{prop}
    Pour trouver $u_n$ connaissant $u_{n+1}$ et $u_{n-1}$, on peut calculer :
\[
u_n = \frac{u_{n+1} + u_n}{2}
\]
Autrement dit, chaque terme est la moyenne arithmétique de ses termes voisins
\end{prop}

\begin{ex}
  \begin{align*}
    u_3 &= \frac{u_4 + u_2}{2}\\
        &= \frac{12 + 6}{2} \\
        &= \frac{18}{2} = 9
  \end{align*}
\end{ex}
\subsection{Suites géométrique}
\label{sec:org7d9102e}

\begin{defi}{Suite géométrique}
On dit qu'une suite est une suite géométrique si chaque terme de la suite est un
multiple du terme précédent par un nombre \(q\).
\end{def}

\begin{ex}
    Si je multiplie par $3$ pour passer d'un terme à un autre à chaque fois, ma suite est géométrique, de raison $3$
\end{ex}

\begin{ex}
\begin{center}
\begin{tabular}{lrrrrr}
\hline
n & 0 & 1 & 2 & 3 & 4\\
\hline
u\textsubscript{n} & 1 & 3 & 9 & 27 & 81\\
\hline
\end{tabular}
\end{center}
\end{ex}
\begin{prop}
    Pour trouver la raison d'une suite géométrique, il suffit de calculer le quotient de deux termes consécutifs
\end{prop}

\begin{ex}
  Dans la suite précédente, on a bien :
  \[
    \frac{u_3}{u_2} = \frac{9}{3} = 3
  \]
\end{ex}

\begin{prop}
    Si $(u_n)$ est une suite géométrique, et que l'on connait $u_{n+1}$ et $u_{n-1}$, alors on peut en déduire la raison de la suite géométrique !
\end{prop}

\begin{ex}
  Si on connait $u_2=9$ et $u_4=81$, puisque l'on a multiplié $u_2$ par $q$ pour trouver $u_3$, et $u_3$ par $q$ pour trouver $u_4$, on peut en déduire que :
  \[
    u_2 \times q \times q = u_4
  \]
  Donc, finalement
  \begin{align*}
    u_2 \times q^2 & = u_4                 \\
    q^{2}          & = \frac{u_{4}}{u_{2}} \\
    q              & = \sqrt{\frac{u_{4}}{u_{2}}} \quad \text{Fonctionnne même si la suite est négative !}
  \end{align*}

  le calcul donne bien :
  \begin{align*}
    q = \sqrt{\frac{81}{9}} = \sqrt{9} = 3
  \end{align*}
\end{ex}

\begin{prop}
Si $(u_n)$ est une suite géométrique, et que l'on connait $u_{n+1}$ et $u_{n-1}$, alors on peut en déduire le terme $u_n$
\end{prop}

\begin{ex}
  Si on reprend l'exemple précédent, on peut s'apercevoir que
  \begin{align*}
    u_3 = \sqrt{u_2 \times u_4}
  \end{align*}
  En effet, on obtient bien :
  \begin{align*}
    u_3 &= \sqrt{9 \times 81} = 3\times 9 = 27
  \end{align*}
\end{ex}

\begin{prop}
  Si $a$ et $b$ sont deux nombres positifs, alors :
  \begin{align*}
    \sqrt{a \times b} = \sqrt{a} \times \sqrt{b}
  \end{align*}
\end{prop}

\begin{ex}
    Voir les calculs faits plus haut.
\end{ex}

\begin{prop}
  On peut calculer directement le $n$-ième terme d'une suite géométrique de raison $q$, par la formule suivante :
  \[
    u_n = u_0 \times q^n
  \]
\end{prop}

\begin{ex}
  Le cinquantième terme de la suite donnée en exemple se calcule par :
  \[
    u_n = 3^n \times 1
  \]
\end{ex}
\end{document}