% Created 2021-09-27 lun. 20:18
% Intended LaTeX compiler: pdflatex
\documentclass[11pt]{article}
\usepackage[utf8]{inputenc}
\usepackage[T1]{fontenc}
\usepackage{graphicx}
\usepackage{grffile}
\usepackage{longtable}
\usepackage{wrapfig}
\usepackage{rotating}
\usepackage[normalem]{ulem}
\usepackage{amsmath}
\usepackage{textcomp}
\usepackage{amssymb}
\usepackage{capt-of}
\usepackage{hyperref}
\usepackage{minted}
\input{/home/fabien/alphabetagammadelta/parametre.tex}
\author{Delhomme Fabien}
\date{\today}
\title{Cours Logarithme}
\hypersetup{
 pdfauthor={Delhomme Fabien},
 pdftitle={Cours Logarithme},
 pdfkeywords={},
 pdfsubject={},
 pdfcreator={Emacs 27.2 (Org mode 9.5)}, 
 pdflang={English}}
\begin{document}

\maketitle

\section{Les puissances}
\label{sec:orgc083207}

\subsection{Definition}
\label{sec:org47be094}

\begin{defi}
On rappelle que la notation $a^{n}$ désigne le calcul :
\begin{align*}
    a^{n} = \underbrace{a \times \ldots \times a}_\text{n fois}
\end{align*}
\end{defi}

\section{La fonction puissance}
\label{sec:org2bca8b3}

\subsection{Definition}
\label{sec:org8154a24}

On peut regarder la fonction puissance de \(10\). Par exemple :

\begin{center}
\begin{tabular}{rr}
nombre de départ & puissance de \(10\) de ce nombre\\
1 & 10\\
1.5 & 31.622\\
2 & 100\\
3 & 1000\\
4 & 10000\\
\ldots{} & \ldots{}\\
x & 10\textsuperscript{x}\\
\end{tabular}
\end{center}

Cette fonction est définie par :
\begin{align*}
    f(x) = 10^{x}
\end{align*}

\begin{rem}
On peut définir des puissances qui ne sont pas entières. Par exemple :
\[
10^{1.5} \approx 31.6227766017
\]
\end{rem}

\begin{rem}
Si vous commencez avec une puissance qui n'est pas entière, vous avez peu de
chance de tomber sur un nombre entier. Ici, on a pris comme puissance $1.5$, et
le résultat, $31,622$ n'est pas entier.
\end{rem}

\section{La fonction logarithme décimale}
\label{sec:org3064e8b}

\begin{defi}{Fonction logarithme}
Cette fonction est la fonction qui «lit» le tableau de la fonction puissance
dans l'autre sens. Cette fonction se note $\log$.
\end{defi}

\begin{center}
\begin{tabular}{rr}
nombre de départ & logarithme décimale de ce nombre\\
\hline
10 & 1\\
31.622 & 1.5\\
100 & 2\\
1000 & 3\\
10000 & 4\\
\ldots{} & \ldots{}\\
\(x\) & \(\log{x}\)\\
\end{tabular}
\end{center}

Donc, on peut en déduire les expressions suivantes :
\begin{align*}
  \log(10)  & = 1 \quad \log(31,622) & \approx 1,5 \\
  \log(100) & = 2 \quad \log(1000)   & = 3
\end{align*}

Le logarithme décimale permet de «mesurer» entre quelle puissance de dix un nombre se trouve. Par exemple :
\begin{align*}
    5 < \log(318327) < 6
\end{align*}
\end{document}