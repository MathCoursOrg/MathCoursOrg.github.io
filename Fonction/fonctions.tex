% Created 2021-09-27 lun. 20:18
% Intended LaTeX compiler: pdflatex
\documentclass[11pt]{article}
\usepackage[utf8]{inputenc}
\usepackage[T1]{fontenc}
\usepackage{graphicx}
\usepackage{grffile}
\usepackage{longtable}
\usepackage{wrapfig}
\usepackage{rotating}
\usepackage[normalem]{ulem}
\usepackage{amsmath}
\usepackage{textcomp}
\usepackage{amssymb}
\usepackage{capt-of}
\usepackage{hyperref}
\usepackage{minted}
\input{/home/fabien/alphabetagammadelta/parametre.tex}
\date{\today}
\title{Fonctions}
\hypersetup{
 pdfauthor={},
 pdftitle={Fonctions},
 pdfkeywords={},
 pdfsubject={},
 pdfcreator={Emacs 27.2 (Org mode 9.5)}, 
 pdflang={English}}
\begin{document}

\maketitle

\section{Disclaimer}
\label{sec:org8cd7bbe}
Le cours suivant ne suit peut-être pas exactement le cours donné en classe.
Néanmoins, le contenu est le même, il n'y a que l'ordre qui est légèrement
changé.

\section{Récapitulatif sur la notion de fonction}
\label{sec:org3e819dc}

On considère des fonctions numériques, c'est-à-dire des fonctions qui associent
à des nombres d'autres nombres.

\subsection{Vocabulaire}
\label{sec:org2332792}

\subsubsection{L'image d'une fonction}
\label{sec:orgabde0e1}

\begin{defi}{Image d'un réel par une fonction}
Soit $x$ un nombre réel. On appelle $f(x)$ l'image du nombre $x$ par la fonction $f$. Autrement dit :
$$f : x \longmapsto f(x)$$
\end{defi}

\begin{rem}
   On désigne les fonctions par des lettres, comme $f$. À l'inverse l'image d'une fonction par un nombre $x$ est notée $f(x)$

   Donc, on fera la distinction entre $f(x)$, qui est un nombre, et $f$, qui désigne une fonction.
\end{rem}


\subsubsection{Antécédent d'une fonction}
\label{sec:org45f8d3d}

\begin{defi}{Antécédent d'une fonction}
L'antécédent de $y$ est un nombre $x$ tel que $f(x) = y$.
\end{defi}

Il faut donc retenir qu'une fonction \textbf{associe} à un nombre son \textbf{image}. Pour une
image donnée, on peut retrouver l'ensemble des nombres qui ont cette image par
une fonction, on les appelle les \textbf{antécédents}.

Schématiquement, on peut donc retenir :
\[
    \textrm{antécédent} \mapsto \textrm{image}
\]

\subsubsection{Courbe représentative}
\label{sec:org4c3b471}

Pour représenter une fonction, on place les antécédents sur la courbe des
abscisses et les images sur l'axe des ordonnées.

 \begin{defi}{Courbe représentative d'une fonction}
On appelle courbe représentative d'une fonction l'ensemble des points du plan
ainsi formé
 \end{defi}

\begin{rem}
Un nombre ne peut avoir qu'une seule image par une fonction, mais une image peut avoir plusieurs antécédents.
\end{rem}
\section{Représentation des fonctions usuelles}
\label{sec:org7b4e92b}

\subsection{Fonctions paires, fonctions impaires}
\label{sec:org5a299aa}

\begin{defi}{Fonction paire}
Une fonction est dite paire si sa courbe représentative présente une symétrie
axiale par rapport à l'axe des ordonnées.
\end{defi}


\begin{defi}{Fonction impaire}
Une fonction est dite impaire si sa courbe représentative présente une symétrie
centrale, de centre l'origine du repère.
\end{defi}

\begin{rem}
Il existe des fonctions qui ne sont ni paires, ni impaires.
\end{rem}
\subsection{La fonction identité}
\label{sec:org4cc68a2}
\begin{defi}{Fonction identité}
 La fonction identité est la fonction qui associe a $x$ le nombre réel $x$. C'est-à-dire qu'elle associe à un nombre ce nombre lui-même.
\end{defi}


\begin{center}
   \begin{tikzpicture}
   \begin{axis}[
       xlabel=$x$,
       ylabel={$f(x) = x$},
       axis lines = center,
   ]
       \addplot[domain=-10:10, samples=100, color=orange]{x};
   \end{axis}
   \end{tikzpicture}
\end{center}

\begin{prop}
  La courbe représentative de la fonction identité est une droite. Cette droite
  passe par l'origine du repère, puisque l'image de $0$ par la fonction identité
  est $0$.

  Tous les points de la courbe représentative de la fonction identité sont de la
  forme $(x; x)$

  La fonction identité est une fonction impaire.
\end{prop}
\subsection{La fonction carré}
\label{sec:orgbe4880f}
\subsubsection{Définition}
\label{sec:orge7f3ddb}
\begin{defi}{Fonction carré}
  La fonction  carre est la fonction qui associe à tout réel $x$ positif
  ou nul le nombre $x^{2}$. Autrement dit,
  \begin{align*}
    f : x \mapsto x^{2}
  \end{align*}
\end{defi}
\subsubsection{Représentation graphique}
\label{sec:org0b5be2c}
\begin{center}
   \begin{tikzpicture}
   \begin{axis}[
       xlabel=$x$,
       ylabel={$f(x) = x^{2}$},
       axis lines = center,
   ]
       \addplot[domain=-10:10, samples=100, color=orange]{x*x};
   \end{axis}
   \end{tikzpicture}
\end{center}
\begin{prop}
  Sa courbe représentative est une parabole. Cette parabole passe par l'origine du
  repère, puisque l'image de $0$ par la fonction identité est $0$.

  Tous les points de la courbe représentative de la fonction identité sont de la forme $(x; x²)$
\end{prop}

\subsection{La fonction cube}
\label{sec:org85f80e6}
\subsubsection{Définition}
\label{sec:orgffb8790}
\begin{defi}{fonction cube}
  La fonction cube est la fonction qui associe à tout réel $x$ positif
  ou nul le nombre $x^{3}$. Autrement dit,
  \begin{align*}
    f : x \mapsto x^{3}
  \end{align*}
\end{defi}
\subsubsection{Représentation graphique}
\label{sec:orgdcc5d09}
\begin{center}
   \begin{tikzpicture}
   \begin{axis}[
       xlabel=$x$,
       ylabel={$f(x) = x^{3}$},
       axis lines = center,
   ]
       \addplot[domain=-10:10, samples=100, color=orange]{x*x*x};
   \end{axis}
   \end{tikzpicture}
\end{center}
\begin{prop}
  Cette courbe passe par l'origine du repère, puisque si on appelle $f$ la
  fonction cube, on a $f(0) = 0$, donc l'image de $0$ par la fonction $f$ est bien $0$.

  Tous les points de la courbe représentative de la fonction cube sont de la forme $(x; x^{3})$

  La fonction cube est une fonction impaire.
\end{prop}

\subsection{La fonction inverse}
\label{sec:orgffc4308}
\subsubsection{Définition}
\label{sec:org3aea505}
\begin{defi}{fonction inverse}
  La fonction inverse est la fonction qui associe à tout réel $x$ positif
  ou nul le nombre $\frac{1}{x}$. Autrement dit,
  \begin{align*}
    f : x \mapsto \frac{1}{x}
  \end{align*}
\end{defi}
\subsubsection{Représentation graphique}
\label{sec:org0eb29b4}

\begin{center}
   \begin{tikzpicture}
   \begin{axis}[
       xlabel=$x$,
       ylabel={$f(x) = \frac{1}{x}$},
       axis lines = center,
   ]
       \addplot[domain=-10:10, samples=100, color=orange]{1/x};
   \end{axis}
   \end{tikzpicture}
\end{center}

 \begin{prop}
  La courbe représentative de la fonction inverse est une hyperbole.

  Cette courbe ne passe par l'origine, puisque l'image de $0$ par cette fonction n'est pas définie.
  Tous les points de la courbe représentative de la fonction inverse sont de la forme
  $(x; \frac{1}{x})$

  La fonction inverse est une fonction impaire.
\end{prop}
\subsection{La fonction racine carrée}
\label{sec:org559ffac}
\subsubsection{Définition}
\label{sec:orgaf2d20a}
\begin{defi}
  La fonction racine carre est la fonction qui associe à tout réel $x$ positif
  ou nul le nombre $\sqrt{x}$. Autrement dit,
  \begin{align*}
    f : x \mapsto \sqrt{x}
  \end{align*}
\end{defi}

\subsubsection{Représentation graphique}
\label{sec:orgf608aa7}
\begin{center}
   \begin{tikzpicture}
   \begin{axis}[
       xlabel=$x$,
       ylabel={$f(x) = \sqrt{x}$},
       axis lines = center,
   ]
       \addplot[domain=-10:10, samples=100, color=orange]{sqrt(x)};
   \end{axis}
   \end{tikzpicture}
\end{center}
\begin{prop}

  La courbe de la fonction racine carrée passe par l'origine du repère, puisque si on appelle $f$ la
  fonction racine carrée, on a $f(0) = 0$.

  Tous les points de la courbe représentative de la fonction racine carrée sont de la forme
  $(x; \sqrt{x})$

  La fonction racine carrée est une fonction qui n'est \textbf{ni paire, ni impaire}.

  La courbe représentative de la fonction racine carrée est exactement la même
  courbe que la demie-parabole de la fonction carré (la partie à gauche des
  ordonnées) qui a subi une rotation de $90$ dans le sens horaire,

\end{prop}

\begin{ex}
\begin{enumerate}
\item \(f(0) = \sqrt{0} = 0\)
\item \(f(4) = \sqrt{4} = 2\) car le carré de \(2\) est \(4\)
\item \(f(2) = \sqrt{2} \approx 1.4142\)
\item \(f(9) = \sqrt{9} = 3\) car le carré de \(3\) est \(9\)
\item \(f(-1)\) \textbf{n'existe pas !} car \(-1\) est plus petit que \(0\)
\end{enumerate}
\end{ex}

\subsection{Les fonctions affines}
\label{sec:org8cd519e}
\subsubsection{Fonctions affines avec coefficient positif}
\label{sec:orgf67fb8e}
\begin{center}
   \begin{tikzpicture}
   \begin{axis}[
       xlabel=$x$,
       ylabel={$f(x) = 3x+2$},
       axis lines = center,
   ]
       \addplot[domain=-10:10, samples=100, color=orange]{3*x+2};
   \end{axis}
   \end{tikzpicture}
\end{center}

\begin{rem}
 La fonction identité est une fonction affine avec un coefficient positif
 particulière (le coefficient est égal à $1$, et l'ordonnée à l'origine est
 nulle).
\end{rem}
\subsubsection{Fonctions affines avec coefficient négatif}
\label{sec:orgbdc6817}
\begin{center}
   \begin{tikzpicture}
   \begin{axis}[
       xlabel=$x$,
       ylabel={$f(x) = -3x+2$},
       axis lines = center,
   ]
       \addplot[domain=-10:10, samples=100, color=orange]{-3*x+2};
   \end{axis}
   \end{tikzpicture}
\end{center}
\begin{prop}
  La courbe représentative des fonctions affines ou linéaires est une droite.
  Inversement, toutes droites dans un rèpère, représente une fonction affine
  (sauf pour les droites qui sont parfaitement verticale, autrement dit
  parallele à l'axe des ordonnées).

  Tous les points de la courbe représentative d'une fonction affine sont de la forme
  $(x; a \times x + b)$

  Les fonctions affines ne sont ni paires, ni impaires dès que $a \not = 0$
\end{prop}
\section{Variations d'une fonction}
\label{sec:orgee66ed5}
\subsection{Intervalle}
\label{sec:orgaf5e214}
Un intervalle désigne un ensemble de nombre réel. On distingue les intervalles
ouverts des intervalles fermés lorsque l'on veut exclure ou non les extrémités.

\subsection{Croissance et décroissance d'une fonction sur un intervalle}
\label{sec:orgbb05f11}
\subsubsection{Définition}
\label{sec:orgc898649}
\subsubsection{Tableau de variation d'une fonction}
\label{sec:orgcf9ecee}
\subsubsection{Application : tableau de variation des fonctions usuelles}
\label{sec:orgf6fbc5f}
\subsubsection{Extremums : définition et inégalité}
\label{sec:org7954d13}
\section{Position relative de courbe}
\label{sec:org69a8c4c}
\section{Résolution d'équation \(f(x) = k\)}
\label{sec:org9ec5ded}
\end{document}