% Created 2021-09-27 lun. 20:18
% Intended LaTeX compiler: pdflatex
\documentclass[11pt]{article}
\usepackage[utf8]{inputenc}
\usepackage[T1]{fontenc}
\usepackage{graphicx}
\usepackage{grffile}
\usepackage{longtable}
\usepackage{wrapfig}
\usepackage{rotating}
\usepackage[normalem]{ulem}
\usepackage{amsmath}
\usepackage{textcomp}
\usepackage{amssymb}
\usepackage{capt-of}
\usepackage{hyperref}
\usepackage{minted}
\input{/home/fabien/Cours/alphabetagammadelta/parametre.tex}
\author{Delhomme Fabien}
\date{\today}
\title{Cours Valeur Absolue}
\hypersetup{
 pdfauthor={Delhomme Fabien},
 pdftitle={Cours Valeur Absolue},
 pdfkeywords={},
 pdfsubject={},
 pdfcreator={Emacs 27.2 (Org mode 9.5)}, 
 pdflang={English}}
\begin{document}

\maketitle

\section{Definition}
\label{sec:org9ca7431}

\subsection{Symbolisme}
\label{sec:org5411d87}
La fonction valeur absolue est une fonction qui s'écrit \(x \mapsto |x|\).

\subsection{Calcul}
\label{sec:org94f6bab}
\begin{defi}{La valeur absolue}
La valeur absolue d'un nombre correspond exactement a sa distance à $0$
\end{defi}

\begin{ex}
La valeur absolue de $-3$ s'écrit $|-3|$, et vaut $3$, puisque $-3$ est à distance $3$ de $0$.
\end{ex}

\section{Représentation graphique}
\label{sec:org0b26480}

Voici à quoi ressemble la représentation graphique de la fonction valeur absolue.
\begin{center}
    \begin{tikzpicture}
    \begin{axis}[
        xlabel=$x$,
        ylabel={$f(x) = |x|$},
        axis lines = center,
    ]
        \addplot[domain=-10:10, samples=100, color=orange]{abs(x)};
    \end{axis}
    \end{tikzpicture}
\end{center}

\section{Lien entre la valeur absolue et la distance entre deux nombres}
\label{sec:org0806dd6}

La valeur absolue d'un nombre représente sa distance avec \(0\).

Mais la valeur absolue de la différence entre deux nombres représente la
distance entre ces deux nombres.

\begin{ex}
  Le nombre $|2 - 3|$ représente la distance entre le nombre $2$ et le nombre $3$, qui vaut donc $1$. Si on fait le calcul, on a :

  \begin{align*}
    |2-3| &= |-1| \\
          &= 1
  \end{align*}

On remarque par ailleurs que $|2 - 3|$ est égal à $|3 - 2|$, puisque la distnace entre $2$ et $3$ est la même que la distance entre $3$ et $2$.
\end{ex}
\end{document}