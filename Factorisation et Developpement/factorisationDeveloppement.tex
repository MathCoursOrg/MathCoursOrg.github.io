% Created 2021-09-27 lun. 20:18
% Intended LaTeX compiler: pdflatex
\documentclass[11pt]{article}
\usepackage[utf8]{inputenc}
\usepackage[T1]{fontenc}
\usepackage{graphicx}
\usepackage{grffile}
\usepackage{longtable}
\usepackage{wrapfig}
\usepackage{rotating}
\usepackage[normalem]{ulem}
\usepackage{amsmath}
\usepackage{textcomp}
\usepackage{amssymb}
\usepackage{capt-of}
\usepackage{hyperref}
\usepackage{minted}
\input{/home/fabien/alphabetagammadelta/parametre.tex}
\author{Delhomme Fabien}
\date{\today}
\title{Factorisation et Developpement}
\hypersetup{
 pdfauthor={Delhomme Fabien},
 pdftitle={Factorisation et Developpement},
 pdfkeywords={},
 pdfsubject={},
 pdfcreator={Emacs 27.2 (Org mode 9.5)}, 
 pdflang={English}}
\begin{document}

\maketitle


\section{Reconnaître une forme factorisée et une forme développée}
\label{sec:org108fd94}

Certains calculs sont sous forme factorisée.

\begin{ex}
L'expression :
\[
(x+7)\times (x - 4)
\]
est sous forme factorisée
\end{ex}

\begin{ex}
    L'expression :
    \[
        x^{2} + 3x
    \]
    n'est pas factorisée, puisque on peut encore trouver des facteurs communs aux différents termes. Cette expression est sous forme développé car il n'y a que des produit de termes simples
\end{ex}

\begin{ex}
   L'expression :
\[
x(x+1) + 1
\]
N'est ni sous forme factorisée (car le terme $1$ n'est pas dans un produit) ni sous forme développée (car il existe le terme $x(x+1)$ qui peut encore être développé
\end{ex}


\section{Développer}
\label{sec:org08c440c}

 \begin{defi}{Développement}
On appelle développement le processus qui permet de passer d'une forme développer à une forme factorisée.
 \end{defi}

\begin{ex}
    \[
        x \times (x + 5) = x^{2} + 5x
    \]
 \end{ex}

\subsection{Techniques pour développer un produit de deux facteurs}
\label{sec:orgc233821}

On peut faire appel à un tableau.

Si on veut développer le produit suivant :
\begin{align*}
(x-4)\times (x + 3)
\end{align*}

on peut utiliser le tableau suivant :

\begin{center}
\begin{tabular}{center}
\hline
\texttimes{} & x & 3\\
\hline
x & x\textsuperscript{2} & 3x\\
-4 & -4x & -12\\
\hline
\end{tabular}
\end{center}

Puis faire la somme de tous les termes du tableau :
 \[
    (x-4)\times (x + 3) =  x^{2} + 3x - 4x -12 = x^{2} -x -12
 \]

\section{Factoriser}
\label{sec:org4305f81}

On appelle la factorisation le processus qui permet de passer d'une forme développée à une forme factorisée

\subsection{Identités remarquables}
\label{sec:org409e073}
Les identités remarquables servent à reconnaitres des expressions pour développer ou factoriser. Il faut donc les connaître dans les deux sens (de gauche à droite du signe égal, mais également de droite à gauche).

\subsubsection{Première identité}
\label{sec:org898e1f9}
\begin{prop}
Pour tout nombres réels $a$ et $b$, on a l'égalité suivante
 \[
    (a + b)^{2} = a^{2} + 2ab + b^{2}
 \]
\end{prop}

\begin{dem}
  \textit{Version graphique}

  \textit{Version littérale}

  On développe le carré par :
  \begin{align*}
    (a+b)^{2} &= (a +b) \times (a +b ) \\
              &= a \times (a +b ) + b \times (a+b)\\
              &= a\times a + a \times b + b \times a + b \times b \\
              &= a^{2} + ab + ba + b^{2} \\
              &= a^{2} + 2 a \times b + b^{2}
  \end{align*}
  En on trouve bien l'égalité annoncée
\end{dem}

\subsubsection{Deuxième identité}
\label{sec:org2ae23af}
En partant de la première identité, on peut en déduire la deuxième facilement.
\begin{prop}
  Pour tout nombre réel $a$, $b$, on a :
  \[
    (a-b)^{2} = a^{2} - 2a \times b + b^{2}
  \]
\end{prop}

\begin{dem}
  Pour montrer cette identité, on peut soit redévelopper comme cela a été fait à
  la démonstration précédente, soit, puisque l'identité précédente est valable
  pour tous les nombres $a$, et $b$, on peut \emph{remplacer} $b$ par $-b$ pour
  obtenir :
  \[
    (a + (-b)) = a^{2} + 2 \times a \times (-b) + (-b)^{2}
  \]
  Sauf que $(-b)^{2} = b^{2}$, donc on obtient
  \[
    (a-b) = a^{2} - 2 \times a \times b + b^{2}
  \]
\end{dem}

\begin{ex}
  On peut par exemple calculer $98^{2}$ en remarquant que $98= 100 - 2$, et donc :
  \[
    98^{2} = (100-2)^{2} = 100^{2} - 2 \times 100 \times 2 + 2^{2}
  \]
  Finalement :
  \[
    98^{2} = 10000 - 400 + 4 =  9604
  \]
\end{ex}

\subsubsection{Troisième identité}
\label{sec:org3b58b7f}
La troisième identité est bien différente des deux autres.
\begin{prop}
  Pour tout $a$ et $b$ réel, on a l'égalité suivante :
  \[
    (a+b)(a-b) = a^{2} - b^{2}
  \]
\end{prop}

\begin{dem}
On démontre cette égalité à l'aide d'un développement comme pour la première identité
\begin{align*}
  (a+b)(a-b) &= a\times (a - b) + b \times (a - b)\\
             &= a\times a - a \times b + b \times a - b \times b \\
             &= a^{2} - b^{2}
\end{align*}
\end{dem}

\begin{rem}
   Qu'obtient-on lorsque l'on remplace $b$ par $-b$, quelle nouvelle identité découvre-t-on  ?
\end{rem}

\section{Applications théoriques : résolution d'équations}
\label{sec:org4705751}
Les identités remarquables permettent de résoudre des équations sans faire
d'erreur. Regardons précisemment l'équations :
\[
    x^{2} = 49
\]

On sait déjà qu'une solution est donnée par \(x = 7\), puisque le carré de \(7\)
fait \(49\). Mais est-ce la seule solution ?

On va utiliser une identité remarquable pour le savoir.

\begin{meth}
  On soustrait $49$ à chaque membre de l'équation, pour obtenir la nouvelle
  équation (équivalente à la première) suivante :

  \(
  x^{2} - 49 = 0
  \)

  On peut
  remplacer 49 par $7^{2}$, pour obtenir :

  \[
    x^{2} - 7^{2} = 0
  \]

  On peut utiliser l'identité remarquable $a^{2} - b^{2} = (a+b)(a-b)$, avec $a =
  x$ et $b = 7$, on obtient donc :

  \(
    (x-7)(x+7) = 0
  \)

  Puis, si un produit est nul, c'est
  que l'un de ses facteurs est nul, donc soit $x$ vaut $7$ (on retrouve une
  solution vue précédemment), soit $x$ vaut $-7$ (et c'est souvent l'equation que
  l'on oublie !)
\end{meth}

\section{Applications concrètes : calculer mentalement des carrés}
\label{sec:orgbe3b97e}

Si on considère le calcul de \(204^{2}\), on peut utiliser la relation suivante :
\begin{align*}
  204^{2} &= (200 + 4)^{2} \\
          &= 200^{2}+2 \times 200 \times 4 + 4^{2}\\
          &=  40000 + 1600 + 16 \\
          &= 41616
\end{align*}

\section{Application concrete et théorique : extraction d'une racine carré}
\label{sec:org43532ea}
\begin{meth}

On vient de voir par exemple que $\sqrt{41616} = 204$. Peut-on calculer à la
main les premieres décimales de $\sqrt{41617}$ ?

On sait que $\sqrt{41617} \approx 204$. Que doit-on rajouter à $204$ pour
s'approcher le plus possible de $\sqrt{41617}$ ?

On pose $h$ le nombre tel que
\begin{align*}
    \sqrt{41617} = 204 + h
\end{align*}
On obtient donc
\begin{align*}
  41617 &= (204 + h)^{2} \\
        &= 204^{2} + 2 \times 204 \times h + h^{2}\\
        &= 41616 + 408 \times h + h^{2}
\end{align*}

Le terme $h^{2}$ devrait être très très petit, donc on va à partir de maintenant
considérer qu'il est nul (alors qu'il ne l'est pas ! Mais on cherche une approximation !)

On se retrouve ainsi avec l'expression :
\begin{align*}
    1 = 408 \times h
\end{align*}
Donc, si on considère que $h^{2}$ est négligeable car très petit, on obtient
finalement $h = \frac{1}{408}$. Il vient :
\begin{align*}
\sqrt{41617} &\approx 204 + \frac{1}{408}
\end{align*}
\end{meth}
\end{document}